% Options for packages loaded elsewhere
\PassOptionsToPackage{unicode}{hyperref}
\PassOptionsToPackage{hyphens}{url}
%
\documentclass[
]{article}
\usepackage{amsmath,amssymb}
\usepackage{iftex}
\ifPDFTeX
  \usepackage[T1]{fontenc}
  \usepackage[utf8]{inputenc}
  \usepackage{textcomp} % provide euro and other symbols
\else % if luatex or xetex
  \usepackage{unicode-math} % this also loads fontspec
  \defaultfontfeatures{Scale=MatchLowercase}
  \defaultfontfeatures[\rmfamily]{Ligatures=TeX,Scale=1}
\fi
\usepackage{lmodern}
\ifPDFTeX\else
  % xetex/luatex font selection
\fi
% Use upquote if available, for straight quotes in verbatim environments
\IfFileExists{upquote.sty}{\usepackage{upquote}}{}
\IfFileExists{microtype.sty}{% use microtype if available
  \usepackage[]{microtype}
  \UseMicrotypeSet[protrusion]{basicmath} % disable protrusion for tt fonts
}{}
\makeatletter
\@ifundefined{KOMAClassName}{% if non-KOMA class
  \IfFileExists{parskip.sty}{%
    \usepackage{parskip}
  }{% else
    \setlength{\parindent}{0pt}
    \setlength{\parskip}{6pt plus 2pt minus 1pt}}
}{% if KOMA class
  \KOMAoptions{parskip=half}}
\makeatother
\usepackage{xcolor}
\usepackage{longtable,booktabs,array}
\usepackage{multirow}
\usepackage{calc} % for calculating minipage widths
% Correct order of tables after \paragraph or \subparagraph
\usepackage{etoolbox}
\makeatletter
\patchcmd\longtable{\par}{\if@noskipsec\mbox{}\fi\par}{}{}
\makeatother
% Allow footnotes in longtable head/foot
\IfFileExists{footnotehyper.sty}{\usepackage{footnotehyper}}{\usepackage{footnote}}
\makesavenoteenv{longtable}
\usepackage{graphicx}
\makeatletter
\def\maxwidth{\ifdim\Gin@nat@width>\linewidth\linewidth\else\Gin@nat@width\fi}
\def\maxheight{\ifdim\Gin@nat@height>\textheight\textheight\else\Gin@nat@height\fi}
\makeatother
% Scale images if necessary, so that they will not overflow the page
% margins by default, and it is still possible to overwrite the defaults
% using explicit options in \includegraphics[width, height, ...]{}
\setkeys{Gin}{width=\maxwidth,height=\maxheight,keepaspectratio}
% Set default figure placement to htbp
\makeatletter
\def\fps@figure{htbp}
\makeatother
\setlength{\emergencystretch}{3em} % prevent overfull lines
\providecommand{\tightlist}{%
  \setlength{\itemsep}{0pt}\setlength{\parskip}{0pt}}
\setcounter{secnumdepth}{-\maxdimen} % remove section numbering
\ifLuaTeX
  \usepackage{selnolig}  % disable illegal ligatures
\fi
\usepackage{bookmark}
\IfFileExists{xurl.sty}{\usepackage{xurl}}{} % add URL line breaks if available
\urlstyle{same}
\hypersetup{
  hidelinks,
  pdfcreator={LaTeX via pandoc}}

\author{}
\date{}

\begin{document}

\section{Title of paper in bold sentence
case}\label{title-of-paper-in-bold-sentence-case}

First I. LastName\textsuperscript{1,2}, First I.
LastName\textsuperscript{2}

\textsuperscript{1}University of Illinois, \textsuperscript{2}Shirley
Ryan AbilityLab (Chicago)

\section{INTRODUCTION}\label{introduction}

This template was created in Microsoft Word. It provides authors with
the formatting specifications needed to prepare their paper. Margins,
column widths, line spacing, font and type styles are built-in: you can
replace the existing text with your own (e.g. copy and paste). Ensure
that the final version follows the guidelines provided in this document.
Do not change the formatting in this template.

All submitted papers should include an ``Introduction'' section, as
shown here, providing background and rationale for the topic addressed.
The introduction section should conclude with a concrete and specific
purpose statement. The remaining sections of the paper should be
appropriate to the content and methodology used, depending upon the type
of paper being submitted. For many research-based papers, this would
include Methods, Results, Discussion and Conclusions; an
Acknowledgements section may also be included. References should be
included as the final section.

\section{SUBMISSION INSTRUCTIONS}\label{submission-instructions}

Submissions must be a minimum of 2 pages and a maximum of 4 pages in
total (i.e. including References). Submissions must be in English.

\section{LAYOUT}\label{layout}

All margins are set at 3/4" (1.9 cm). Use an 11-point sans serif font
(Aptos, Calibri, or Arial) for all text. All text (except the title)
should be single-spaced and aligned left (not justified). A 6-point
space should follow each paragraph and heading; do not indent
paragraphs. The title should be centered, bold font and sentence case
(i.e. only the first word capitalized).

Two versions of the paper are required: a complete version and a blinded
version. The complete version should be named according to the title of
the paper (e.g
``Measuring\_heart\_rate\_in\_manual\_wheelchair\_users.doc'').

You can shorten this name if your title is very long, but do not include
the author's name. In the blinded version, delete the author(s) names
and affiliations, as well as any identifying information. Ensure that
the paper has no review comments visible that might reveal who the
author is (i.e. accept or delete all track changes/comments). The
blinded version should also be named according to the title of the
paper, but preceded by the word ``Blinded'' (e.g.
``Blinded\_Measuring\_heart\_rate\_in\_manual\_wheelchair\_users.doc'').

\section{HEADINGS}\label{headings}

Main headings (e.g. Introduction, Methods, Results etc.) should be
written in bold font and all capital letters. Sub-headings may be used,
but only if there are at least two sections within that level of
content. First level sub-headings are written in bold font and sentence
case. Second level sub-headings, if required, are underlined and written
in sentence case, as illustrated below. Do not indent headings or
sub-headings.

\subsection{\texorpdfstring{\textbf{Example of a first level
sub-heading}}{Example of a first level sub-heading}}\label{example-of-a-first-level-sub-heading}

\subsection{Example of a second level
sub-heading}\label{example-of-a-second-level-sub-heading}

\section{FIGURES AND TABLES}\label{figures-and-tables}

Figures should be labeled with the figure number and title/description
in bold font, directly below the figure. Font within a figure should not
be smaller than 8-point. The figure should be edited so that the main
text wraps around the image (i.e. Format \textless{} Shape \textless{}
Layout \textless{} Square). A text box frame is a simple way to insert
images: use Insert \textgreater{} Text Box; size the text box and then
drag images into the box. To remove the visible box, select the frame
and select Format \textgreater{} Text Box \textgreater{} Colors and
Lines \textgreater{} No Fill and \textgreater{} No Line.

\includegraphics[width=2.19463in,height=0.43458in]{./media/media/image10.png}

\textbf{Figure 1. The RESNA logo}

Table headings should appear immediately above the table in bold font,
aligned left, and include the table number. Table headings and
sub-headings should be bold font and centered relative to the row and
column. Table content/text should be regular font and aligned left. Font
within the table should be no smaller than 8-point.

\textbf{Table 1. Example of a table heading}

\begin{longtable}[]{@{}
  >{\raggedright\arraybackslash}p{(\columnwidth - 4\tabcolsep) * \real{0.2656}}
  >{\raggedright\arraybackslash}p{(\columnwidth - 4\tabcolsep) * \real{0.3537}}
  >{\raggedright\arraybackslash}p{(\columnwidth - 4\tabcolsep) * \real{0.3807}}@{}}
\toprule\noalign{}
\endhead
\bottomrule\noalign{}
\endlastfoot
\multirow{2}{=}{Table Column Heading} &
\multicolumn{2}{>{\raggedright\arraybackslash}p{(\columnwidth - 4\tabcolsep) * \real{0.7344} + 2\tabcolsep}@{}}{%
Table Column Heading} \\
& Table Subheading & Table Subheading \\
Table text & Table text & Table text \\
Table text & 85.5\% & 14.5\% \\
\end{longtable}

\section{}\label{section}

\section{}\label{section-1}

\section{}\label{section-2}

\section{}\label{section-3}

\section{REFERENCES}\label{references}

In-text citations must be numbered using square brackets as indicated
here. {[}1{]} Multiple sources are cited in the following manner.
{[}2,3,4-7{]} References may be presented in either APA or Vancouver
style format, but must be listed in order of citation and preceded by
the citation number in square brackets. Each reference is followed by a
6-point space. Some examples of reference formats are provided below:

Journal articles:

{[}1{]} Jutai JW, Rigby P, Ryan S, Stickel S. Pyschosocial impact of
electronic aids to daily living. Assist Technol. 2000 12(2):123--131.

Book:

{[}2{]} Portney LG, Watkins MP. Foundations of clinical research:
applications to practice. 3\textsuperscript{rd} ed. Philadelphia: F.A.
Davis Company; 2015.

Chapter in a Book:

{[}3{]} Pagel JF, Pegram GV. The role for the primary care physician in
sleep medicine. In: Pagel JF, Pandi-Perumal SR, editors. Primary care
sleep medicine. 2nd ed. New York: Springer; 2014.

Conference Proceeding

{[}4{]} Smith RO. Assistive technology outcome assessment prototypes:
measuring INGO variables of outcomes. In R. Simpson (Ed.), Proceedings
of the RESNA 25th Annual Conference (239--341). Minneapolis: RESNA
Press; 2002.

\end{document}
